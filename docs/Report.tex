\documentclass[10pt]{amsart}

\usepackage[utf8]{inputenc}

\usepackage[spanish]{babel}
\usepackage{blindtext}

\usepackage{amsmath}
\usepackage{amssymb}
\usepackage{amsfonts}
\usepackage{color}
\usepackage{hyperref}
\usepackage{url}
\usepackage{stmaryrd}
\usepackage{calrsfs}
\usepackage{fancyhdr}
\usepackage{textcomp}
\usepackage{graphicx}
\usepackage{stmaryrd}
\usepackage{lipsum}

\voffset=-1.4mm
\oddsidemargin=14pt
\evensidemargin=14pt
\topmargin=26pt
\headheight=9pt     
\textheight=576pt
\textwidth=441pt %441
\parskip=0pt plus 4pt

\pagestyle{headings}
\title{Proyecto de Programaci\'on Funcional en Haskell}
\author{Programaci\'on Declarativa}
\date{\today}

\begin{document}
	\begin{titlepage}
		\clearpage
		\maketitle
		
		\vspace{3em}
		\begin{center}
			Tema: \textbf{Generador y solucionador de hidatos sudoku} 	

            \vspace{6em}
			\begin{center}
        		\includegraphics[width=6cm]{haskell.png}
        	\end{center}

			\vspace{6em}
			Autores: \\
			Leandro Rodríguez Llosa \\
			Andry Rosquet Rodríguez
		\end{center}
		\thispagestyle{empty}
	\end{titlepage}

    \normalsize
 
	\small
	\section*{Iniciador}
	,..................Leo aqui pon el iniciador de todo o sea el main o algo asi.................. 
	\section*{ ¿C\'omo se genera el sudoku?}
	\subsection*{Acercamiento general}
	Para la construcci\'on de un tablero con \'unica soluci\'on primero es necesario establecer las dimensiones de este y adem\'as fijar 1 y el valor del n\'umero m\'as alto posible de dicho tablero. Lo anterior a la par que se crea otra matriz con iguales dimensiones que brinda la informaci\'on correspondiente a las posiciones donde es posible ubicar alg\'un valor del sudoku (True) y con False las que no son v\'alidas, llam\'emosle m\'ascara. Luego se busca una posible soluci\'on y teniendo el tablero resuelto se procede a analizar qu\'e valores son obligatorios en la posici\'on que este se encuentra y los que son prescindibles. Teniendo los valores fijados y algunos espacios en blanco, tendr\'iamos un tablero con soluci\'on \'unica.
	\subsection*{M\'etodos y estrategias}
	El m\'etodo que establece la forma del tablero inicial es ..... Escribe aqui leo con lo que estabas haciendo tu de generar el tablero inicail con random y el tema...........................
	Luego teniendo el tablero con una soluci\'on v\'alida, se procede a utilizar la funci\'on \textit{findUniqueSolution} la cual utiliza dicho sudoku y su m\'ascara booleana para encontrar un nuevo tablero con espacios vac\'ios que tenga soluci\'on \'unica. Dicho m\'etodo itera del 2 al m\'aximo posible del tablero, procediendo a realizarle un an\'alisis a cada uno en relaci\'on con su ubicaci\'on en el sudoku resuelto. ¿En qu\'e consiste dicho an\'alisis?, se procede a remover dicho n\'umero del tablero y se hace un llamado a la funci\'on \textit{solveNumber} la cual devuelve el n\'umero de soluciones posibles del tablero resultante. Esto conllevar\'ia a dos posibles resultados que son explicados a continuaci\'on:
	 Si el resultado es 1 se elimina dicho valor de manera definitiva porque significa que dicho d\'igito siempre va a, necesariamente, ser ubicado en esta posici\'on.
	 Si el resultado es mayor que 1 esto implica que dicho d\'igito debe estar ubicado obligatoriamente en esta posici\'on, de lo contrario si no estuviese fijado generar\'ia m\'ultiples posibles soluciones.
	 Cuando se realiz\'o dicho an\'alisis a cada valor, se obtendr\'ia el tablero con soluci\'on \'unica donde estar\'an n\'umeros fijados si son necesarios y espacios disponible que solo brindar\'ian soluci\'on v\'alida si se ubican los n\'umeros requeridos.
	 Las funciones anteriores se auxilian de otros m\'etodos para funcionar:
	 
    \begin{itemize}
        \item \textit{changeState} que cambia un tablero, ubicando un nuevo valor en una ubicaci\'on dada.
        \item \textit{correctlyPlaced} recibiendo un valor x que ya se encuentra ubicado en el tablero y la posici\'on del antecesor a este brindar\'a True si est\'an adyacentes y False de lo contrario.
        \item \textit{giveNumber} ser\'a el iterador por cada n\'umero, el cual solo se detiene si llega al n\'umero m\'aximo del tablero, de lo contrario llama a \textit{validPlace} con cada una de las posibles ubicaciones adyacentes a el antecesor. Finalmente este devolver\'a el n\'umero de soluciones del tablero recibido, sumando la cantidad de soluciones resultantes al ubicar cada n\'umero en cada ubicaci\'on valida.  
		\item \textit{validPlace} se encarga de comprobar si una posici\'on dada es v\'alida seg\'un la m\'ascara por medio de \textit{notValidMask} y si la posici\'on no se encuentra ya ocupada con \textit{notValidHidato}. Si cumple todos los requisitos entonces se hace un llamado 'a \textit{giveNumber} con el sucesor. Lo anterior asegura un ciclo continuo hasta que se haya recorrido cada n\'umero.
    \end{itemize}
	 
      Se utilizaron otras funciones como \textit{searchNumber}, \textit{searchNumber}, \textit{searchTop}, \textit{getRows} y otros que su comportamiento responden al trabajo con el tablero y otorgan informaci\'on requerida en cierto momento del proyecto en general.
	 
    \section*{¿C\'omo se soluciona el tablero?}
	 
    \subsection*{Acercamiento general}
	Recibiendo un tablero con valores fijados, el 1 y el m\'aximo entre ellos, y la m\'ascara correspondiente a dicho sudoku, el m\'etodo solve se encarga de darle soluci\'on. Similar al generador esta funci\'on itera desde 2 al m\'aximo pero en este caso se analiza que ocurre si se ubica el n\'umero en cada posici\'on adyacente a su antecesor, comprobando siempre que sea v\'alida, y luego se avanza a comprobar el sucesor.
	 
    \subsection*{M\'etodos y estrategias}
	El iterador solo se detendr\'a cuando se compruebe que el antecesor al m\'aximo se encuentra bien ubicado, una vez haya ocurrido esto significa que tenemos la soluci\'on, de lo contrario se continua iterando por las dem\'as posiblidades.
	
    Se utiliza ubica, que es la encargada de iterar por todos los n\'umeros y analizar que ocurre si este se fija en cada posici\'on valida adyacente al antecesor por medio de la funci\'on \textit{validPlace}. 
    
    \end{itemize}
     El solucionador utiliza al igual que el generador funciones como \textit{changeState}, \textit{correctlyPlaced}, \textit{searchTop}, \textit
	\textit{searchNumber}, \textit{getRows}, \textit{getColumns}, \textit{notValidMask}, \textit{notValidHidato} y otros con el mismo uso.	  



 
	\section*{Generador}
	\subsection*{generateHidato :: Hidato -> IO Hidato}

 
	\subsection*{ii}
	NOOOOOO
 

	\subsection*{Direction}
	
\end{document}